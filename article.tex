\documentclass[a4paper, 12pt]{report}
\usepackage[14pt]{extsizes} % для того чтобы задать нестандартный 14-ый размер шрифта
\usepackage[utf8]{inputenc}
\usepackage[russian]{babel}
\usepackage{setspace,amsmath}
\usepackage[left=20mm, top=15mm, right=15mm, bottom=15mm, nohead, footskip=10mm]{geometry} % настройки полей документа

\usepackage{geometry}
\usepackage{tikz}
\usetikzlibrary{calc}
\usetikzlibrary{decorations.pathmorphing}
 
\begin{document} % начало документа
 
\begin{titlepage}

\begin{tikzpicture} [overlay,remember picture]
    \draw [line width=0.5mm ] 
		($ (current page.north west) + (1cm, -1cm) $)
    rectangle
    ($ (current page.south east) + (-1cm, 1cm) $);
\end{tikzpicture}

% НАЧАЛО ТИТУЛЬНОГО ЛИСТА
\begin{center}

\hfill \break
\large{Министерство образования РФ}\\
\large{Московский физико - технический институт}\\
\normalsize{2020 - 2021 учебный год}\\ 
\hfill \break
\hfill \break
\normalsize{Физтех - школа радиотехники и компьютерных технологи}\\
\hfill \break
\hfill \break
\normalsize{Реферат по теме:}\\
\hfill \break
\LARGE{«Культура и быт языческой Руси»}\\

\end{center}

\hfill \break

\vspace*{4cm}
\begin{flushright}
  Выполнил: \\
  студент 1 курса \\
  группы Б01-003 \\
  Лепарский Роман Денисович \\
  \hfill \break
  Проверил: \\
  Доц. Департамента истории \\
  Черникова Л.П. 
\end{flushright}
 
\vspace*{4cm}
\begin{center}
	г. Москва (Долгопрудный) \\
	2020 г.
\end{center}

\thispagestyle{empty} % выключаем отображение номера для этой страницы
 
% КОНЕЦ ТИТУЛЬНОГО ЛИСТА

\restoregeometry

\end{titlepage}

\setcounter{page}{2}
 
\newpage
     
    \tableofcontents % Вывод содержания
\newpage
 
\newpage

%===========================================================================================================================================

\chapter{Введение}

Влюбой стране культура развивается в тесном взаимоотношении с общественными изменениями и потрясениями. Культурное развитие имеет тесную
взаимосвязь с историей, и это естественно касается и русской культуры. Н. Бердяев – знаменитый русский философ – в истории русской культуры
выделял пять основных периодов\footnote{Россия киевская, времени татарского ига, московская, петровская, советская}, одним из которых является период языческой культуры Древней Руси. Это первый и самый древний слой культуры России, хронология которого относится к середине первого тысячелетия после Рождества Христова.

\section{Проблематика} 



\section{fse}
 
\end{document}  % КОНЕЦ ДОКУМЕНТА !